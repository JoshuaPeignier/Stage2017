\documentclass{beamer}

\usepackage[utf8]{inputenc}
\usepackage{default}
\usepackage[T1]{fontenc}%
\usepackage{amssymb}
\usepackage{array}
\usepackage{amsmath}
\usepackage{graphicx}
\usepackage[english]{babel}
\usepackage{pgf,pgfplots}
\pgfplotsset{compat=1.8}
\usetheme{Berlin}
\useoutertheme{split}
\definecolor{grenat}{RGB}{139,26,26}
\definecolor{beige}{RGB}{255,255,220}

\def\E {{\mathbb E}}
\def\EE {{\mathbb E}}
\def\FF {{\mathbb F}}
\def\II {{\mathbb I}}
\def\LL {{\mathbb L}}
\def\NN {{\mathbb N}}
\def\PP {{\mathbb P}}
\def\RR {{\mathbb R}}
\def\ZZ {{\mathbb Z}}
\def\RRpos {{\mathbb R_{\geq 0}}}
\def\RRposi {{\mathbb R_{\geq 0}^{\infty}}}
\def\Max {\textnormal {Max}}
\def\Min {\textnormal {Min}}
\def\Sup {\textnormal {Sup}}
\def\Inf {\textnormal {Inf}}

\title{\textbf{Possibility Distribution Semantics for Probabilistic Programs with Nondeterminism}}
\author{Joshua Peignier}
\date{15th May 2017 - 11th August 2017}

\setbeamercolor{structure}{fg = grenat}
\setbeamercolor{background canvas}{bg = beige}

\defbeamertemplate*{footline}{shadow theme}
{%
  \leavevmode%
  \hbox{\begin{beamercolorbox}[wd=.5\paperwidth,ht=2.5ex,dp=1.125ex,leftskip=.3cm plus1fil,rightskip=.3cm]{author in head/foot}%
    \usebeamerfont{author in head/foot}\insertframenumber\,/\,\inserttotalframenumber\hfill\insertshortauthor
  \end{beamercolorbox}%
  \begin{beamercolorbox}[wd=.5\paperwidth,ht=2.5ex,dp=1.125ex,leftskip=.3cm,rightskip=.3cm plus1fil]{title in head/foot}%
    \usebeamerfont{title in head/foot}\insertshorttitle%
  \end{beamercolorbox}}%
  \vskip0pt%
}

\setbeamertemplate{blocks}[rounded]%
[shadow=false]

\begin{document}
\frame{\titlepage}




\section{Introduction}
%Explain here what are the probabilistic and nondeterministic choices
%Explain why they are used:
% - Probabilistic choices: allows resolution of problems with algorithms having a better average complexity
% - Nondeterministic choices: used in model checking to model the unpredictable behavior of some programs -> Not meant to be executed
% Give an example of a program combining both:
% What does it do ?
\begin{frame}{Plan}
  \tableofcontents[sectionstyle=show/shaded]%,subsectionstyle=show/shaded/hide ]
\end{frame}

\begin{frame}
\frametitle{Probabilistic programs}
What are probabilistic programs ?
\begin{itemize}
\item<2-> Programs including probabilistic choice
\item<3-> For each input, returns a distribution of outputs
\item<4-> Widely used to solve problems with a better average complexity 
% Mention the different examples: quicksort, Freivalds' matrix multiplication, or primality tests
\end{itemize}
\onslide<5->{
\begin{block}{Example}
$$\{ x := -1 \} [\frac{1}{3}] \{x := 1\}$$
Simulates the flipping of a biased coin.
\end{block}
}
\end{frame}

\begin{frame}
\frametitle{Nondeterministic programs}
What are nondeterministic programs ?
\begin{itemize}
\item<2-> Programs including nondeterministic choice
\item<3-> Not a concept meant to be executed: only exists for model checking purposes
\item<4-> Represents the unpredictable behavior of a program.
%Explain in details that it does not make any sens to talk about probabilities here, because, it only models possible outcomes, for which we have no more information than the very fact that they can happen.
\end{itemize}
\onslide<5->{
\begin{block}{Example}
$$\{ x := -1 \} \square \{x := 1\}$$
\onslide<6->{Not the same as probabilistic choice: no probabilities here.}
\end{block}
}
\end{frame}

\begin{frame}
\frametitle{Problem}

\begin{block}{Problem}
\onslide<2-> Defining proper semantics for programs containing both\newline
\onslide<3->{$\rightarrow$ Complicated}
\end{block}

\onslide<4->{
\begin{block}{Example}
\onslide<5->{$$P_0\textnormal{: } \{ \{ x := 2 \} \square \{ x := 5 \} \} [p] \{ x := 7 \}$$}
\onslide<6->{\begin{center} How can we semantically describe this program ? \end{center}}
\end{block}
}
\end{frame}

\section{Problem}
\begin{frame}{Plan}
  \tableofcontents[sectionstyle=show/shaded]%,subsectionstyle=show/shaded/hide ]
\end{frame}
%Explain that there exists clean semantics, for nondeterministic programs, but not for program including both ? Or at least, there exists some, but they are not clean.

\begin{frame}
\frametitle{Semantics for nondeterministic programs}
\begin{itemize}
\item<1-> For non-deterministic programs: weakest precondition calculus.
\item<2-> Dijkstra: "programs are viewed as predicate transformers"
\item<3-> $\varphi$ postcondition, $P$ program\newline
\onslide<4-> {$\rightarrow wp[P](\varphi)$: \emph{weakest precondition}}
\end{itemize}
\end{frame}

\begin{frame}
\frametitle{Semantics for nondeterministic programs}
\begin{block}{Example}
\onslide<1->{$$P_1\textnormal{: } x := -y ; x := x+1$$}
\onslide<2->{$$\varphi = [x \geq 5]$$}
\begin{align*}
\onslide<3->{wp[P_1](\varphi)} &\onslide<4->{= wp[x := -y ; x := x+1]([x \geq 5])} \\
&\onslide<5->{= wp[x := -y]([x + 1 \geq 5])} \\
&\onslide<6->{= wp[x := -y]([x \geq 4]) }\\
&\onslide<7->{= [-y  \geq 4]}\\
&\onslide<8->{= [y \leq -4]}
\end{align*}
\end{block}
\end{frame}

\begin{frame}
\frametitle{Semantics for probabilistic nondeterministic programs}
\begin{itemize}
\item<1-> There exists an extension for probabilistic nondeterministic programs
\item<2-> $wp[P](\varphi)$: probability that $\varphi$ is satisfied after thee execution of $P$
\item<3-> Problem: not possible to assign probabilities to nondeterministic choices\newline
\onslide<4->{$\rightarrow$ Conventions must be chosen}
\end{itemize}
\end{frame}


% Quickly talk about the demonic choice, explain that "we consider that the outcome which is realised is the least desired one
% Explaine that the results are counterintuitive for two reasons: they violate the modularity law of probabilities, and we are missing the information that x can take the value 2 (to realise it, we must compute the wp of another postcondition)
\begin{frame}
\frametitle{Semantics for nondeterministic programs}
\begin{block}{Example}
\onslide<1->{$$P_0\textnormal{: } \{\{x := 2 \} \square \{ x := 5\} \} [p] \{x := 7\}$$}
\begin{align*}
\onslide<3->{wp[P_0]([x = 7]) =} & \onslide<4->{1-p} \\
\onslide<5->{wp[P_0]([x = 2]) =} & \onslide<6->{0} \\
\onslide<7->{wp[P_0]([x = 5]) =} & \onslide<8->{0} \\
\onslide<9->{wp[P_0]([x = 2 \textnormal{ OR } x = 5]) =} & \onslide<10->{p} \\
\end{align*}
\end{block}
\end{frame}

% Explain that possibilities, as the name indicates, represent how possible an event is, not how likely it is to happen.
\begin{frame}
\frametitle{Possibility-based semantics}
\begin{itemize}
\item<1-> Idea: rather use possibility based semantics
\item<2-> Possibility measure: $\Pi(U \cup V) = \Max(\Pi(U),\Pi(V))$
\end{itemize}
\end{frame}

\section{The previous semantics}

% Explain the idea of anticipation of values, show the problems
% Do not detail the rules
\begin{frame}
\frametitle{test}
test
\end{frame}

\section{Our semantics}

% Show our own semantics, explain the difference : instead of anticipating values of real-valued functions, we anticipate values of set-valued functions, where the sets represent possible values
% Apply to P_0, see how it behaves
\begin{frame}
\frametitle{test}
test
\end{frame}

\section{Conclusion}

% Explain that our rules are not proven to make sense for programs containing loops, because we need a complete partial order...
% This is a future work
\begin{frame}
\frametitle{Conclusion}
\begin{itemize}
\item<1-> Original problem: existing semantics for probabilistic nondeterministic programs\newline
\onslide<2->{$\rightarrow$ Imprecise, miss information}
\item<3-> Our idea: correct this lack of information with possibility theorey
\item<4-> New rules, our semantics describes the behavior of some programs better than the existing semantics
\item<5-> Main problem: not proven to be well-defined for program with rules
\item<6-> Future work: find a complete partial order $\rightarrow$ Then, our semantics will be well-defined
\end{itemize}
\end{frame}

\end{document}
